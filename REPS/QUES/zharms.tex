% Created 2023-05-16 Tue 10:26
% Intended LaTeX compiler: pdflatex
\documentclass[11pt]{article}
\usepackage[utf8]{inputenc}
\usepackage[T1]{fontenc}
\usepackage{graphicx}
\usepackage{longtable}
\usepackage{wrapfig}
\usepackage{rotating}
\usepackage[normalem]{ulem}
\usepackage{amsmath}
\usepackage{amssymb}
\usepackage{capt-of}
\usepackage{hyperref}
\usepackage{tcolorbox}
\usepackage{minted}
\usepackage[margin=1in]{geometry}
\usepackage{xcolor}
\author{Nidish Narayanaa Balaji}
\date{\today}
\title{Balancing zero harmonics in the Wave-Based Formalism}
\hypersetup{
 pdfauthor={Nidish Narayanaa Balaji},
 pdftitle={Balancing zero harmonics in the Wave-Based Formalism},
 pdfkeywords={},
 pdfsubject={},
 pdfcreator={Emacs 29.0.90 (Org mode 9.6.3)}, 
 pdflang={English}}
\begin{document}

\maketitle
I will take the example of the Euler Bernoulli Beam whose dispersion relationship is given by
\begin{equation}
\label{eq:org5c8404b}
k = {\left( \frac{\rho A_r}{E_y I_y} w^2 \right)}^{1/4}.
\end{equation}
The solution is expanded through the wave-based formalism as,
\begin{equation}
\label{eq:org12cf09e}
u(x,t) = (a_1^+ e^{kx} + a_1^- e^{-kx} + a_2^+ e^{ikx} + a_2^- e^{-ikx}) e^{-i\omega t} + c.c.,
\end{equation}
where \(a_1^{\pm}\) and \(a_2^{\pm}\) are the evanescent and traveling wave components respectively. 

Considering a joint between two points, say \(A\) and \(B\), with relative displacement denoted by \(\delta u(x,t)\rvert_{x=x_J}\), the joint force balance equation can be written as,
\begin{equation}
\label{eq:org802ae02}
V(x,t)\rvert_{x=x_J} = -E_y I_y \frac{\partial^3 u}{\partial x^3} = f_{joint} (\delta u ),
\end{equation}
where \(f_{joint}(\cdot)\) is the joint constitutive model, \(E_yI_y\) is the flexural rigidity, and \(V(x,t)\rvert_{x=x_J}\) is the shear force at the joint.

Substituting the wave-based solution ansatz from eq. \ref{eq:org12cf09e} into eq. \ref{eq:org802ae02} yields,
\begin{equation}
\label{eq:org3f11885}
-E_y I_y k(\omega)^3 \begin{bmatrix} 1 & -1 & -i & i & 0 & 0 & 0 & 0\\
1 & -1 & -i & i & -1 & 1 & i & -i \end{bmatrix}
\begin{bmatrix} a_1^+\\ a_1^-\\ a_2^+\\ a_2^-\\ b_1^+\\ b_1^-\\ b_2^+\\ b_2^- \end{bmatrix} =
\begin{bmatrix} f_{joint}(\delta u)\\ 0 \end{bmatrix}.
\end{equation}
In the above vector equation, the first row equation represents the force balance from eq. \ref{eq:org802ae02} and the second row equation represents the fact that the interface force is identical across the joints.

\begin{tcolorbox}[title={Question},colback=red!5!white,colframe=red!75!black, colbacktitle=yellow!50!red,coltitle=red!25!black, fonttitle=\bfseries,subtitle style={boxrule=0.4pt, colback=yellow!50!red!25!white}]
When \(\omega\) is set to zero, \(k(\omega)=0\) (see eq. \ref{eq:org5c8404b}). As a consequence, eq. \ref{eq:org3f11885} represents a null expression (matrix in lhs multiplied by zero). In other words, this \textbf{requires} the static component of the joint force to be zero. Does this make sense or is something wrong? \textbf{What if I have a joint that provides a static force? How can I represent this in this formalism?}

This also leads to the more general question of \textbf{can the wave based approach accommodate constant terms in the PDE?}

\end{tcolorbox}
\end{document}